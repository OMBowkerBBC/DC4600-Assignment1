\section{Design}
  In this section I have designed activity and sequence diagrams for some of the proposed changes. I have colour coded where possible to show
  errors/failures and bad routes, tending to be in red, successful routes in green, and the yellow which signifies a potential error, or
  inaccuracy within the system but not a total failure.

  \subsection{Adding a new user}
    As part of adding a new user I have also modelled the login/register, I did this because without being signed in a customer
    would not be able to rent.

    \begin{figure}[H]
      \centering
      \includegraphics[width=12cm]{diagrams/activity/Activity1.png}
      \caption{Activity diagram for adding a new user, this includes sign in/up.}
      \label{fig:newUserActivity}
    \end{figure}

    The above diagram shows 3 happy paths and 2 sad paths. The sad paths are the user quitting the process, however it is worth
    noting that a user could quit the application during any stage of the process. The first happy path is when a sessions exists.
    This points to some kind of session/cookie management in the future. The second is credentials stored in the database of an 
    existing user.

    \begin{figure}[H]
      \centering
      \includegraphics[width=12cm]{diagrams/sequence/Sequence1.png}
      \caption{Sequence diagram for adding a new user, this includes sign in/up.}
      \label{fig:newUserSequence}
    \end{figure}

    In the above sequence diagram a user can technically exit the login/register process at any stage, however I did not map this in the 
    diagram as every stage would become bloated. Both EmailHandler and CredentialsManager would be module of
    the system, however I have separated them for more clarity. License validator would be an external API Call. 
  
  \newpage
  
  \subsection{Taking a payment}

    \begin{figure}[H]
      \centering
      \includegraphics[width=12cm]{diagrams/activity/Activity2.png}
      \caption{Activity diagram for taking a payment.}
      \label{fig:takePaymentActivity}
    \end{figure}

    Taking a payment refers to online payments. If the user fails multiple times (3) then the system should stop
    the user from trying again, at least for a short while. This is to stop spamming and potential malicious activity.
    This is also includes the path of paying with crypto and cards to help reach some of Aston's business goals.

    \begin{figure}[H]
      \centering
      \includegraphics[width=12cm]{diagrams/sequence/Sequence2.png}
      \caption{Sequence diagram for taking a payment.}
      \label{fig:takePaymentSequence}
    \end{figure}

    The user navigates to the payment screen and is shown details about rental by th UI. Then the system consistently
    loops to get the user payment. This broken when the payment succeeds or the systems payment failure
    limit is reached (shown in the dark red). This diagram introduces the CryptoWallet entity which the system will have
    to interact with if it wants to accept cryptocurrency payments.

  \newpage

  \subsection{Adding a new car}

    \begin{figure}[H]
      \centering
      \includegraphics[width=6cm]{diagrams/activity/Activity5.png}
      \caption{Activity diagram for adding a new car to the system.}
      \label{fig:newCarActivity}
    \end{figure}

    When adding a car a user has two optional stages, taxing the vehicle and MOT'ing the car. These are needed for the 
    car to be rentable, however don't necessarily have to be done when adding the vehicle. However the diagram shows that
    if these criteria are not met the car is marked as unrentable until they have been fulfilled.

    \begin{figure}[H]
      \centering
      \includegraphics[width=12cm]{diagrams/sequence/Sequence5.png}
      \caption{Sequence diagram for adding a new car to the system.}
      \label{fig:newCarSequence}
    \end{figure}

    The actors/entities in this diagram are similar to the first. In this case however I have a boundary (UI) that controls
    what the users is presented with. The DVLA in this case would be some sort of API call, or potentially a physical call a
    member of staff has to make. For the sake of ease in this case I have modelled it as an API to check both 
    tax and MOT status of the vehicle. The car is marked as unrentable if either of these cases are not fulfilled and that
    would be a simple flag in the database.

\newpage