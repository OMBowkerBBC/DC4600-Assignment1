\section{Overview}

  Currently ACME uses an outdated paper-based approach, communicating with customers and other staff members using a combination of telephone and email.
  The current system has numerous problems such as: 
  \begin{enumerate}
    \item Due to the paper-based approach important details such as details about orders, cars and accounting can often go missing causing issues 
    for both customers and internal staff.
    \item Backups of data are difficult and time consuming due to the paper-based approach and their risk management for the same reason is near non-existent.
    \item Customer experience is not optimal due to the updates coming through telephone calls or email only.
  \end{enumerate}

  In addition to this, the system does not contribute to ACMEs current business goals, which are:
  \begin{enumerate}
    \item \textbf{Increase profits/customers -} ACME has being seeing a decrease in profits and customers and wants to increase these.
    \item \textbf{Improve documentation resilience and navigability -} Documents often go missing and are hard to find, by moving away from a paper based
    system ACME hopes to make this issue less of a problem.
    \item \textbf{Cater to the student demographic -} ACME wants to take advantage of the large student population in the area, with both Aston and Birmingham
    university being near by. They are willing to offer better deals to students and offer alternative payment methods to target this demographic.
    \item \textbf{Automate/speed up time intensive tasks -} Due to the current system, data input is slow and taking and editing rentals is also slow. With a 
    new system ACME hopes this will speed things up, potentially requiring less staff, which will also help increase their profits.
  \end{enumerate}

  \subsection{Proposed changes}
  In order to fulfill the above business goals ACME has decided to upgrade it's outdated system with a new automated, digital system. The new system will no 
  longer use paper based records, instead opting for a digital solution, which can either be in the cloud or an on prem server. The solution should not remove 
  any of the current functionality of the system. However can replace them for more modern alternatives. Some of the planned replacements include:
  \begin{enumerate}
    \item A new system where users can sign up and book rentals, without direct interaction from staff members. This includes a new payment system where
    customers will be able to pay through the new application immediately without going into the store. The details of both customers and order will be
    stored in the new database. These changes will help ACME to achieve business goals 1, 3 and 4.

    \item A new system for staff to add/edit/delete cars in the system, these details will also be stored in a database. This will help to reach business goals
    2 and 3, by speeding up internal workings due to switching to a database instead of the old paper based system. It may also indirectly help with business goal 1
    as staff will have more time to do more important things for the company. Another potential is that the number of staff needed could be reduced due to the 
    optimisation, however this would need to be though about due to potential ethical issues. 
    
    \item As part of increasing profits and catering to the student demographic, ACME has made a bold plan to try an incorporate cryptocurrency payments
    into its new system. Cryptocurrency adoption in the UK has been growing in popularity, doubling since 2019 [1]. In addition to this a survey done in Germany 
    showed that '18-to 27-year old survey respondents were three times more likely to own a digital currency' [2] and BanklessTimes wrote an article summarising
    a Finder report that showed 38\% of all cryptocurrency holders in the UK were between the ages of 18-34 [3].

    \begin{table}[H]
      \centering
      \begin{tabular}{|l|l|}
        \hline
        Age Group & Adoption Rate \\ \hline
        18-34     & 38\%          \\ \hline
        34-54     & 43.5\%        \\ \hline
        55+       & 20\%  \\ \hline     
      \end{tabular}
      \caption{Table showing adoption rate based on age in the UK [3]}
    \end{table}
  \end{enumerate}

  \subsection{Software development model}
  The software development lifecycle (SDLC) usually consists of between 5-7 phases. This isn't a strict rule however with names changing and 
  certain phases often not being included. The figure below shows the SDLC I will be following, \hyperref[sec:AppendixA]{\textbf{Appendix A}} 
  shows a full 7 stage model.

  \begin{figure}[H]
    \centering
    \includegraphics[width=12cm]{assets/sdlc.png}
    \caption{Figure showing the phases of SDLC. [4]}
    \label{fig:sdlc}
  \end{figure}

  \begin{enumerate}
    \item \textbf{Plan -} The planning phase involves, what the project is going to be, the requirements of the project, identifying stakeholders and 
    any feasibility studies that need to be done.
    \item \textbf{Design -} This phase can include graphical UI/UX designs, but also designs of the software. This can be done using UML.
    \item \textbf{Develop -} Write the code that the plan 
    \item \textbf{Test -} Write tests for the written code and perform manual testing.
    \item \textbf{Deploy -} Deploy to cloud/on prem infrastructure.
    \item \textbf{Review -} Review the new features/changes added and start the cycle again.
  \end{enumerate}

  This report covers only covers the plan and some design aspects of the SDLC. Using an SDLC includes benefits such as helping understand requirements,
  identify risks [6] and speed up delivery of a project. Imagine skipping the plan and design phase above, jumping straight into development. The developers
  would not know what the system should look like and deliver a subpar final product

  For this project I would recommend the use of the agile framework. This methodology is described as:
  \begin{quote}
    \textit{'The Agile methodology is a project management approach that involves breaking the project into phases and emphasizes continuous collaboration and 
    improvement. Teams follow a cycle of planning, executing, and evaluating.'} [7]
  \end{quote}

  By this description our phase mappings would be, Planning = (Plan, Design), Executing = (Develop, Test and Deploy) and Evaluating = (Review). The agile manifesto
  describes the key concerns of agile.

  \begin{figure}[H]
    \centering
    \includegraphics[width=12cm]{assets/agileManifesto.png}
    \caption{Agile methodology key principles [8]}
    \label{fig:agileManifesto}
  \end{figure}

  Business goals 1 and 3 lend themselves to a more agile approach. Approaches such as waterfall can be \textit{'largely dependent upon how much work is done 
  upfront, especially research'} [9]. This research can take a lot of time, and with new students starting in 3 months, a large opportunity could be missed.
  This point is also linked to the agile manifestos 'Working software over comprehensive documentation' idea. In addition to this, ACME wants to be gain its
  old customer base back, agile encourages 'Customer collaboration'. Anthill reported that  \textit{'71\% of customers feel frustrated when an experience is 
  impersonal or company focussed.'} [10]. By interfacing with the customer/user of the application ACME could build good will with it's customer as well as 
  make a better product for the target audience in the long run.
\newpage