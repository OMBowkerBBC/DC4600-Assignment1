\section{Overview}

Currently ACME uses an outdated paper-based approach, communicating with customers and other staff members using a combination of telephone and email.
The current system has numerous problems such as: 
\begin{enumerate}
  \item Due to the paper-based approach important details such as details about orders, cars and accounting can often go missing causing issues 
  for both customers and internal staff.
  \item Backups of data are difficult and time consuming due to the paper-based approach and their risk management for the same reason is near non-existent.
  \item Customer experience is not optimal due to the updates coming through telephone calls or email only.
\end{enumerate}

In addition to this, the system does not contribute to ACMEs current business goals, which are:
\begin{enumerate}
  \item \textbf{Increase profits/customers -}
  \item \textbf{Improve documentation resilience and navigability -}
  \item \textbf{Cater to the student demographic -}
  \item \textbf{Automate/speed up time intensive tasks -}
\end{enumerate}

\subsection{The Plan}
In order to fulfill the above business goals ACME has decided to upgrade it's outdated system with a new automated, digital system. The new system will no 
longer use paper based records, instead opting for a digital solution this can either be in the cloud or an on prem server. The solution should not remove 
any of the current functionality of the system. However can replace them for more modern alternatives. Some of the planned replacements include:
\begin{enumerate}
  \item A new system where users can sign up and book rentals, without direct interaction from staff members. This includes a new payment system where
  customers will be able to pay through the new application immediately without going into the store. The details of both customers and order will be
  stored in the new database. These changes will help ACME to achieve business goals 1, 3 and 4.

  \item A new system for staff to add/edit/delete cars in the system, these details will also be stored in a database. This will help to reach business goals
  2 and 3, by speeding up internal workings due to switching to a database instead of the old paper based system. It may also indirectly help with business goal 1
  as staff will have more time to do more important things for the company. Another potential is that the number of staff needed could be reduced due to the 
  optimisation, however this would need to be though about due to potential ethical issues. 
  
  \item As part of increasing profits and catering to the student demographic more, ACME has made a bold plan to try an incorporate cryptocurrency payments
  into its new system. Cryptocurrency adoption in the UK has been growing in popularity, doubling since 2019 [1]. In addition to this a survey done in Germany 
  showed that '18-to 27-year old survey respondents were three times more likely to own a digital currency' [2] and BanklessTimes wrote an article summarising
  a Finder report that showed 38\% of all cryptocurrency holders in the UK were between the ages of 18-34 [3].

  \begin{table}[H]
    \centering
    \begin{tabular}{|l|l|}
      \hline
      Age Group & Adoption Rate \\ \hline
      18-34     & 38\%          \\ \hline
      34-54     & 43.5\%        \\ \hline
      55+       & 20\%  \\ \hline     
    \end{tabular}
    \caption{Table showing adoption rate based on age in the UK [3]}
  \end{table}
\end{enumerate}

\newpage