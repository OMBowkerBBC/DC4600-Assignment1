\section{Feasibility of project}
  The new system would be a large shift in how ACME works, for that reason I will now briefly discuss the feasibility of the changes in different areas
  of the business. There will inevitably be overlap between these areas.
  \vspace{0.2cm}

  \noindent\textbf{Economic -} As previously mentioned ACME's sales have been declining which makes money something to heavily consider with the proposed
  changes. They're not in the 'red' however and can afford some wiggle room. The main issue right now is that they don't have the technical staff to create 
  these new system. Glassdoor puts the average software developer at £43,110 a year, £59,137 for a senior [17]. In my opinion ACME could probably hire a senior and
  two 'mid-level' engineers to do this project, adding up to around £140,000 a year. This is a lot of money, however the new system in the long run could result
  in less jobs needed in other areas that the system now controls, meaning total outgoing could stabilise. Another option would be to hire contractors, however
  these are more costly, and would need to be called in anytime anything went wrong with system.
  
  In addition to salaries there is also the cost of running the new system. I would recommend a cloud solution instead of a physical server. Not only would the
  upfront cost be less, you also only pay for what you use. They can also provide deals to companies who use their systems to save a bit more money on top.
  
  \vspace{0.2cm}

  \noindent\textbf{Technical -} Thanks to use of the agile framework we can take advantage of something called a spike:
  \begin{quote}
    \textit{'Spike A story or task aimed at answering a question or gathering information, rather than at producing shippable product.'} [18]
  \end{quote}
  This would allow the new developers to pinpoint any threats that aren't seen in the planning and requirements gathering beforehand. This spike code can technical
  be used to help build the final MVP. In terms of can the proposed system be created? The main system is a CRUD [18] application
  and should be very simple to build. It may be worth hiring a security analyst at the end of the project to make sure that the system is secure for both users 
  and staff.
  The hardest part of the project would be to include cryptocurrency payments. Providers such as MetaMask [20] provide easy to integrate with
  wallets. However confirming the payment could be a little more challenging, however there are multiple libraries that can help with this. For this issue I would
  recommend that at least one developer has some experience with blockchain development, otherwise the learning curve could cost time and money due to training.
  When it comes to final implementation frameworks such as React Native [21], or Flutter [22] can be used for cross device/platform availability, thus speeding up the
  total development time.
  \vspace{0.2cm}

  \noindent\textbf{Operational -} The new changes may not be welcomed by some staff. This could be because they fear the change or feel like their job is at risk.
  Where possible ACME should tackle this by keeping an open line of communication with current staff and offering training to current staff for new roles.

  As well as staff issues there are potential regulatory issues when it comes to holding personal user data and cryptocurrency as whole. with the latter, it
  does seem like some progress is being made in a positive direction. The FCA (Financial Conduct Authority) released a report that seemed very positive
  towards crypto in the future [23]. In regards to holding user details such as driving license and banking information, there are system like Stripe [15] 
  (for finance) that has a fantastic API [24] that means this data will never be stored by ACME. Driver license also does not need to be stored locally and can be 
  checked and be a simple flag in the database.

\newpage