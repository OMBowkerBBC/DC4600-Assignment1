\section{Stakeholders and Use Cases}
  Before writing requirements it's important to know the stakeholders of the company, as well as who will be affected by the changes that are planned.

  \subsection{Identifying stakeholders}
    To help identify stakeholders I have created a viewpoint diagram that can be seen in the below figure.
    \begin{figure}[H]
      \centering
      \includegraphics[width=12cm]{diagrams/viewpoints.drawio.png}
      \caption{Viewpoint diagram to help identify stakeholders.}
      \label{fig:viewpoint}
    \end{figure}

    I have broken down the stakeholders into 3 categories, External, Interactors and Indirect, which describe their
    exposure to the new systems.

    \vspace{0.2cm}

    \noindent\textbf{External -} This group of people will not part of ACMEs staff, however do have an interest/role in the system/company. Insurance, 
    and both stakeholders under 'Authorities' are needed for the system to function, insurance for the cars, and authorities to
    check the details of a car and renter are valid. The payment handler, I give the example of Stripe [15] will also be a part of the system
    but will not directly interact with it. Finally shareholders of a company always have an interest in change due to the expectation of profit.

    \vspace{0.2cm}

    \noindent\textbf{Interactors -} Interactors are people who will use the new system directly. I have separated these out into staff and customers.
    Customer is separated into two smaller classes, student and non-student. I did this to reflect one of the business goals of ACME, which is to \textit{
    cater to the student demographic}, in addition to this stakeholder potentially getting benefits due to their situation. The other branch of interactors
    is staff. These are the people within the company who will interact with the new systems, like adding cars to the database (Customer Service Representative)
    and being able to easily query order/financial information (Finance/Accounting).

    \vspace{0.2cm}

    \noindent\textbf{Indirect -} These are employees of ACME who will not directly interact with the new interfaces. Their jobs will stay the same as prior. Despite
    this I though it a good idea to mention them as they could be the target for future upgrades to systems. In addition to this the due to the increased 
    productivity of the new system, a mechanic may be expected to do more work, which may not be feasible.

  \subsection{Use case diagrams}
    Taking the above stakeholders I have created a use case diagram to \textit{'describe a system's requirements strictly from the outside looking in;
    they specify the value that the system delivers to users'} [14]. I will discuss each actor individually here but a full
    use case diagram can be seen in \hyperref[sec:AppendixB]{\textbf{Appendix B}}, including crossovers between actors.

    \subsubsection{Customer Service Representative}
      The CSR refers to the member of staff that interfaces with the customer. They will be able to do all the same activities they did before, now using
      the updated system. I have decided to continue to allow telephone bookings and enquiries to be made. Hopefully the new system will be
      adopted by most people, however for people who aren't comfortable with it still have this option as  
      \textit{'21\% of Britain's population lack basic digital skills'} [16].
      \begin{figure}[H]
        \centering
        \includegraphics[width=10cm]{diagrams/useCase/csr.png}
        \caption{Use case diagram for customer service representative.}
        \label{fig:UCcsr}
      \end{figure}

    \subsubsection{Customer}
      I merged students into a more general category of customer, although they may be able to do different actions in the future that is not the current 
      scope of the changes. The customer can still extend, cancel and pay for rentals, as well as of course renting and returning a car.
      They will now also be able to login/register on the new system, meaning they will no longer have to give their details multiple times. They will also 
      be able to browse all available cars for rental on the new system.  
      \begin{figure}[H]
        \centering
        \includegraphics[width=10cm]{diagrams/useCase/customer.png}
        \caption{Use case diagram for customer.}
        \label{fig:UCcustomer}
      \end{figure}
    
    \subsubsection{Finance/Accounting}
      The finance/accounting department are responsible for generating tax information. This has been an issue in the past where
      receipts and order details have gone missing. The new system hopes to fix this by allowing all this data to be inputted into 
      a database. Finance/accounting will be able generate earning reports and profits and loss statements from this.
      \begin{figure}[H]
        \centering
        \includegraphics[width=10cm]{diagrams/useCase/finance.png}
        \caption{Use case diagram for finance/accounting department.}
        \label{fig:UCaccounting}
      \end{figure}

    \subsubsection{Insurance Company}
      The insurance company is an external stakeholder that ACME staff with have to interact with. They are responsible for insuring the cars and
      are linked to the Customer Service Representative in this task.
      \begin{figure}[H]
        \centering
        \includegraphics[width=10cm]{diagrams/useCase/insurance.png}
        \caption{Use case diagram for insurance company.}
        \label{fig:UCinsurance}
      \end{figure}
      
    \subsubsection{Mechanic}
      The mechanic is in charge of making sure the cars are suitable to rent/drive. Mechanics can service a car, and repair cars
      when needed. Sometimes parts will be ordered when repairing a car, so an optional action has been added to generate a receipt
      for these costs. This would currently be done paper based, however this could be rolled into the new system to make all expenses/profits 
      stored in the same place, making it easier to do things such as tax returns and profit/loss accounts.
      \begin{figure}[H]
        \centering
        \includegraphics[width=10cm]{diagrams/useCase/mechanic.png}
        \caption{Use case diagram for mechanic.}
        \label{fig:UCmechanic}
      \end{figure}

    \subsubsection{Payment Service (Stripe)}
      The payment service is solely responsible for handling payments. They can take a payment for an order and they can
      validate a payment. When validating a payment there is a chance that it fails, therefore they must inform us if this
      happens.
      \begin{figure}[H]
        \centering
        \includegraphics[width=10cm]{diagrams/useCase/paymentService.png}
        \caption{Use case diagram for payment service provider.}
        \label{fig:UCpayment}
      \end{figure}
\newpage