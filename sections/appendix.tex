\section{Appendix}

  \label{sec:AppendixA}
  \subsection{Appendix A - An example of a 7 phased SDLC}
    \begin{figure}[H]
      \centering
      \includegraphics[width=12cm]{assets/sdlc7.jpg}
      \caption{SDLC with 7 phases [5]}
      \label{fig:SDLC7}
    \end{figure}
  
  \newpage

  \begin{landscape}
    \label{sec:AppendixB}
    \subsection{Appendix B - Use case diagram with all actors}
      \begin{figure}[H]
        \centering
        \includegraphics[width=20cm]{diagrams/useCase/fullUseCase.png}
        \caption{Full use case diagram, showing all actors.}
        \label{fig:UCfull}
      \end{figure}
  \end{landscape}
  \newpage

  \subsection{Appendix C - Diagram descriptions}
    \label{sec:AppendixC}

    \subsubsection{Adding customer information}
      As part of adding a information I have also modelled the login/register, I did this because without being signed in a customer
      would not be able to rent.

      \textbf{Activity -} The diagram shows 3 happy paths and 2 sad paths. The sad paths are the user quitting the process, however it is worth
      noting that a user could quit the application during any stage of the process. The first happy path is when a sessions exists.
      This points to some kind of session/cookie management in the future. The second is credentials stored in the database of an 
      existing user.

      \noindent\textbf{Sequence -} In the sequence diagram a user can technically exit the login/register process at any stage, however I did not map this in the 
      diagram as every stage would become bloated. Both EmailHandler and CredentialsManager would be module of the system, however I have separated them for 
      more clarity. License validator would be an external API Call. 

    \subsubsection{Taking payment}
      \textbf{Activity -} Taking a payment refers to online payments. If the user fails multiple times (3) then the system should stop
      the user from trying again, at least for a short while. This is to stop spamming and potential malicious activity.
      This is also includes the path of paying with crypto and cards to help reach some of Aston's business goals.

      \noindent\textbf{Sequence -} The user navigates to the payment screen and is shown details about rental by th UI. Then the system consistently
      loops to get the user payment. This is broken when the payment succeeds or the systems payment failure limit is reached (shown in the dark red). 
      This diagram introduces the CryptoWallet entity which the system will have to interact with if it wants to accept cryptocurrency payments.

    \subsubsection{Handling the return of a vehicle}
      \textbf{Activity -} If the user returns the car with no issues then this process is very simple and the car is marked as available for other 
      customers to rent. If the car does have problems, and the customer doesn't have accidental damage cover, then they will be charged for the 
      damages to the vehicle. I have marked this path as red, as it is not the ideal path, however technically the vehicle has been returned.

      \noindent\textbf{Sequence -} The sequence diagrams shows how staff communicate in person as well as what the system does. The keys are stored
      in a safe on the car lot premises for security. The mechanic checks the vehicle for issues on return, order parts from a supplier and generates 
      a receipt of costs if needed. The system in this case is mainly used to store data in the new database, and send a receipt to the customer at the 
      end if there was any damages they need to pay for.

    \subsubsection{Starting new hire}
      \textbf{Activity -} When a starting a hire the customer must be at the car lot. I have modelled a situation where the customer does not have
      a valid rental and they can then be offered an in person choice of available vehicles. This will help capture people who just walk in ad hoc,
      and help accommodate a different sales path. I have also added a path where the customer can purchase damage cover. This would cover the
      individual if they were in an accident that was partly their doing, so standard insurance wouldn't cover it. However this is not required to 
      start a hire.

      \noindent\textbf{Sequence -} The sequence diagram follows the activity diagram closely. The system communicates with the database and payment
      provider to achieve the desired outcome. Note that I have not fully modelled customer payment transactions as that was already done in the
      \hyperref[sec:design:payment]{\textbf{Taking Payment}} section of this report. There is one fail case and that is if the customer does not have
      a valid rental and does not wish to rent any vehicles that are available.

    \subsubsection{Adding new car to the system}
      \textbf{Activity -} When adding a car a user has two optional stages, taxing the vehicle and MOT'ing the car. These are needed for the 
      car to be rentable, however don't necessarily have to be done when adding the vehicle. However the diagram shows that if these criteria are not 
      met the car is marked as unrentable until they have been fulfilled.

      \noindent\textbf{Sequence -} The actors/entities in this diagram are similar to the first. The DVLA in this case would be some sort of API call, 
      or potentially a physical call a member of staff has to make. For the sake of ease in this case I have modelled it as an API to check both tax and 
      MOT status of the vehicle. The car is marked as unrentable if either of these cases are not fulfilled and that would be a simple flag in the database.
  
  \newpage

  \subsection{Appendix D - Word count}
    \label{sec:AppendixD}
    Word count \textbf{(Number Here)} - includes removal of captions, contents, list of figures/tables, references and appendix.